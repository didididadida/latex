\documentclass[cn,11pt]{elegantbook}

%插入图片包
\usepackage[graphicx]{realboxes}
\usepackage{subfigure}
\usepackage{float}



\title{我的世界}
\author{S.K.}
\institute{蹈微}
\date{October 1, 2023}
\cover{1.jpg}


\begin{document}



\maketitle
\frontmatter
\tableofcontents%命令输出论文目录
\mainmatter

\chapter{前奏}

我的世界不大,也就只是收揽了我所看到的世界。

在之前的讨论中,我自己从心智成长、哲学的宇宙、自然世界、人类世界、
元素世界、数学世界、物理世界、网络世界……中曾经大概遨游了一番。

就我愚钝的感知,其实好多问题都会在网络世界中得到答复。


\section{学贯东西}

古人谈创新,是面向整个社会,今人谈创新,大多是一些局部的改变,看起来今人的创新,总是好像经不起社会的考量。
实则不然,只有现象达到一定程度,才能克服社会的惯性。

在个人的零零总总的世界里,总是需要好多的要素,达到了同频共振,才有一个新的个体,可以说是诞生。
而我这具几乎要尘埃落定的身体,也在过去的不远的时光里,才算是在某些事情上自我完成了一致的认知。

\textcolor{blue}{自认为如果要有所成就,不得不往更深,更广处探索。}

\section{圣人之道}

圣人之道,吾性自足,不假外求。王阳明如此,我又何尝不可以如此呢。


在西南大学度过的一天里,我想到的是:写字时广阅群书,力求统一。

\textcolor{blue}{凝心聚神,是人做任何事情的基础。}

\section{从今天起}
目前有着各种思想和知识围绕着你,需要的是有一个统一的纲领式的文本。
用以指导个人的人生轨迹。

永安眠

谁在京陵画永长,谁在琅琊临书彷?

不争此昼数寸光,问时间安以何往。

\section[short]{人生苦短,就学python}



\appendix
\chapter{自然资料}

\href{https://www.yuque.com/jibenqiejiangdeguawazi/twzv0m/go91tr}{\underline{语雀:我的世界}}

\end{document}
